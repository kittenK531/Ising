\documentclass[a4paper]{article}
\usepackage{amssymb,amsmath}
\usepackage{bm}
\usepackage{parskip}
\usepackage{fancyhdr}
\usepackage{tabu}
\usepackage{enumerate}
\usepackage{graphicx}
\usepackage{caption}
\usepackage{subcaption}
\usepackage{tikz}
\usepackage{siunitx}
\usepackage{multirow}
\usepackage{hyperref}
\usepackage{dcolumn}
\usepackage{hhline}
\graphicspath{{/Users/kei/Desktop/4061/ising/figs/}}

\addtolength{\oddsidemargin}{-.875in}
	\addtolength{\evensidemargin}{-.875in}
		\addtolength{\textwidth}{1.75in}
			\addtolength{\topmargin}{-.875in}
				\addtolength{\textheight}{1.75in}

				\author{Name: Chan Ying}
				\date{\today}
				\begin{document}
				\newcolumntype{d}[1]{D{.}{.}{#1}}
				\title{ \bf Proposal and progess documentation of 2D Ising model development}
				\maketitle

				\section{Motivation}
				This is a developing code aiming to calculate the average observable values of ising model motivated by the high degree of freedoms from statistical physics.

				\section{Theory}
				\subsection{Statistical mechanics: Basics}
				Consider a lattice with \textbf{set}  $\Lambda$ lattice sites. For each lattice site $k\in\Lambda$, the discrete variable $\sigma$ can take $\sigma_k \in \{+1, -1\}$.

				A spin configuration can also be expressed as $\sigma = \{\sigma_k\}_{k\in \Lambda}$.

				\section{Method}

                \end{document}